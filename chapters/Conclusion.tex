\chapter*{Conclusion} \label{Conclusion}
\addcontentsline{toc}{chapter}{Conclusion}

Quantum computational chemistry is likely to become one of the most researched applications of quantum computing. This could help to solve classically intractable chemistry and materials science problems and study phenomena like superconductivity, solid-state physics, transition metal catalysis and certain biochemical reactions. In turn, this increased understanding may help to refine, and perhaps even one day design, new compounds of scientific and industrial importance. \\
This advantage, however, will require quantum computers with high computational power, i.e. with logical qubits, high-fidelity connectivity and low gate time. Capability that will be achieved in a varying period of time, strictly depending on major breakthroughs in engineering of quantum processors. \\
\\
Throughout the thesis we have given a brief introduction to quantum computing and we have discussed the main algorithms developed for classical and quantum devices to solve the electronic structure problem. We have analyzed those from the mathematical and the computational point of view and studied their technological requirements. \\
Finally, we have focused on the conditions to define a computational advantage and on the targets that could be researched to achieve it and we have provided a forecast of applicability for industrially relevant systems. \\
\\
Future works on this topic should focus on performing more experiments to provide more and more precise estimates on computational requirements. Some frameworks for running simulations are already available and easily accessible, among the most popular are Qiskit-nature, PennyLane and tequila \cite{Kottmann2020Nov}. These could provide a more widely spread research on this topic, thus it is important to also develop them to collect more experimental results on ever-growing devices. \\
Although a fundamental portion of research must be dedicated to finding better techniques to make high-fidelity and scalable quantum processors, it is important that even mathematical and software algorithms are improved and optimized, i.e. better state preparation and orbitals representation, better quantum measurement process and hamiltonian simulation with a better scaling.