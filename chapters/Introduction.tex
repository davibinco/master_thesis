\chapter*{Introduction} \label{Introduction}
\addcontentsline{toc}{chapter}{Introduction}

Computer simulation is a fundamental part of most scientific research. It allows to predict the behaviour of a real-world system by creating a mathematical model on a computer and then analyzing the obtained results, without the need to perform a real experiment until one wants to compare the accuracy of the model used. Here we want to analyze the simulation of quantum systems, i.e. physical systems that follow the laws of quantum mechanics, also referred to as 'quantum simulation' or 'computational chemistry'. \\
From the birth of the theory at the beginning of the twentieth century many models have been proposed, and following the terminology of computer science we also refer to them as methods or algorithms. The main problem these algorithms aim to solve is to find the energy of the ground state of the system, this lets us know the most stable configuration of the system, thus the most probable configuration in which we find it in a real experiment. \\
\\
This field of research intersects the investigation of new technological devices on which one can perform such simulation. As a matter of fact, in the last forty years there has been a great development of the so-called quantum computers, a type of calculators that exploit the laws of quantum mechanics to perform the computations. Notoriously, physicist Richard Feynman explained this possibility in a 1982 speech \cite{Feynman1982Jun}, where he referred explicitly to "an exact simulation, that the computer will do exactly the same as nature", where the system under study and the calculator, i.e. the system through which we perform the computation, share the same properties and are in a one-to-one correspondence. \\
This investigation also translates in engineering problems, in effect quantum systems are difficult to manipulate, they need to be isolated and put in specific conditions to work properly, e.g. superconducting circuits need near absolute zero temperatures. For this reason, only in recent years quantum computers with very few components have been produced and are available for use. \\
\\
Through this thesis we explore the main possibilities to run a quantum simulation. We analyze the methods developed for classical computers and the ones for quantum computers doing a comparison with regard to computational requirements and accuracy of the calculations. \\
In Chapter \ref{Quantum computational chemistry} we explore the importance of quantum simulation and the scheme to implement one. We then describe the first historical methods used to perform a simulation on classical computers and the birth of the research for methods on quantum computers, also called 'quantum computational chemistry'. \\
In Chapter \ref{Density functional theory} we describe density functional theory, the main algorithm used today to perform a quantum simulation on a classical computer. \\
Then we focus on quantum computers, in Chapter \ref{Quantum computing} we show the characteristics of this kind of devices and the new potential they bring. We then concentrate on fault-tolerant quantum computing (FTQC) devices, and the methods designed for them, in Chapter \ref{Quantum computing for computational chemistry: FTQC devices}. These are theoretical universal computers which take full advantage of the potential of quantum computation. After this, in Chapter \ref{Quantum computing for computational chemistry: NISQ devices} we characterize noisy intermediate-scale quantum (NISQ) devices and the correspondent methods, these are already available computers with a little number of low quality components, called 'qubits', to perform calculations. \\
Finally in Chapter \ref{Computational advantage} we describe the features of the already available quantum computers from a technological perspective, we do a comparison between the different algorithms and present a forecast of the targets of simulation, whitin chemistry and materials science, that can show an advantage and their computational cost.